% Trabalho de Estudos do Meio Físico
%
% Abaixo seguem orientações originais do modelo utilizado.
%
% ==============================================================
%
% Modelo para monografia de final de curso, em conformidade
% com normas da ABNT implementadas pelo projeto abntex2.
%
% Este arquivo é fortemente baseado em exemplo distribuído no
% mesmo projeto. O projeto abntex2 pode ser acessado pela página
% http://abntex2.googlecode.com/
%
% Este arquivo pode ser rodado tanto com o pdflatex quanto com
% o lualatex.  Como contém referências bibliográficas a serem
% processadas pelo programa bibtex, este programa deve ser
% executado. Em resumo, a ordem de execução deve ser:
% rodar primeiro o pdflatex (ou o lualatex), depois o bibtex e,
% a seguir, o pdflatex (ou o lualatex ) novamente mais duas vezes,
% para assegurar que todas as referências bibliográficas e 
% citações estejam atualizadas.
%
% Para adaptar os textos para uso pessoal, usar os comandos
% imediatamente antes do \begin{document} (iniciando com o
% comando \titulo).  
%
% Este modelo está adaptado para monografias de final de curso
% em matemática da UFRJ, mas, com o uso das variáveis, pode ser
% usado para outros tipos de trabalho (mestrado, doutorado),
% outros cursos, universidades etc.  Caso a adaptação das
% variáveis não seja suficiente, pode-se alterar os comandos
% imprimircapa, imprimirfolhaderosto e imprimiraprovação, 
% fazendo as alterações necessárias.  Como os comandos definidos
% neste texto usam somente LaTeX, a sua adaptação deve ser 
% simples, bastando algum conhecimento de LaTeX.
%
% O restante do preâmbulo provavelmente  não necessitará ser
% alterado, a menos, eventualmente, das opções de chamada da
% classe abntex2, que estão definidas a seguir.
% 
\documentclass[ 
% -- opções da classe memoir que é a classe base da abntex2 --
% tamanho da fonte
12pt,
% capítulos começam em pág ímpar. Insere pág vazia, se preciso
openright,
% para imprimir uma página por folha ou visualização em video 
oneside,
% frente e verso. Margens das pag. ímpares diferem das pares.
%  twoside,
% tamanho do papel. 
a4paper,
% Caio - Ocultando bordas horríveis em hiperligações
hidelinks,
% -- opções da classe abntex2 --
% títulos de capítulos convertidos em letras maiúsculas
%  chapter=TITLE,
% títulos de seções convertidos em letras maiúsculas
%  section=TITLE,
% títulos de subseções convertidos em letras maiúsculas
%  subsection=TITLE,
% títulos de subsubseções convertidos em letras maiúsculas
%  subsubsection=TITLE,
% -- opções do pacote babel --
english,   % idioma adicional para hifenização
portuguese,   % o último idioma é o principal do documento
oldfontcommands,
]{abntex2}
%
% ==============================================================
%
% --------------------------------------------------------------
% Adicionando pacotes para recursos adicionais e defindo opções
% pertinentes
% --------------------------------------------------------------
%
% cabeçalho comum para uso com lualatex ou pdflatex
\usepackage{ifluatex}
% opções para uso com o lualatex
\ifluatex
\usepackage{fontspec}
\defaultfontfeatures{Ligatures=TeX}
% o fonte small caps é diferente no latin modern
\fontspec[SmallCapsFont={Latin Modern Roman Caps}]{Latin Modern Roman}
% pacotes da AMS 
\usepackage{amsmath,amsthm} 
% pacote para fonte específico para símbolos matemáticos
\usepackage{unicode-math}
\setmathfont{Latin Modern Math}
% latin modern tem simbolos de mathbb muito feios.
%  Trocar o fonte para estes simbolos.
\setmathfont[range=\mathbb]{Tex Gyre Pagella Math}
% opções para uso com o pdflatex
\else
\usepackage[utf8x]{inputenc}
\usepackage[T1]{fontenc}
\usepackage{lmodern}
\usepackage{etoolbox}
% pacotes da AMS 
\usepackage{amsmath,amssymb,amsthm} 
% Mapear caracteres especiais no PDF
\usepackage{cmap}
\fi

% pacotes usados tanto pelo lualatex quanto pelo pdflatex
\usepackage{lastpage}    % Usado pela Ficha catalográfica
\usepackage{indentfirst} % Indenta primeiro parágrafo 
\usepackage{color}       % Controle das cores
\usepackage{graphicx}    % Inclusão de gráficos
\usepackage{wrapfig}     % gráficos ao redor do texto
% pacote para ajustar os fontes em cada linha de forma a
% respeitar as margens
\usepackage{microtype}
% permite a gravação de texto em um arquivo indicado a partir
% deste arquivo.  Originalmente foi usado para criar o arquivo
% .bib com conteúdo de exemplo, evitando a edição de um arquivo
% .bib somente para a bibliografia
\usepackage{filecontents}

% Caio - diagramas
% http://www.texample.net/tikz/examples/smart-priority/
%\usepackage{smartdiagram}

% Caio - ladeando imagens
% https://tex.stackexchange.com/questions/57433/cannot-use-caption-under-minipage
\usepackage{caption}

% Caio - preciso de tabelas longas
% http://www.tex.ac.uk/FAQ-figurehere.html
\usepackage{longtable}

% Caio - quero alternar as cores das linhas da tabela
% https://tex.stackexchange.com/questions/107944/alternate-row-colors-in-longtable
\usepackage[table]{xcolor}
\definecolor{lightgray}{gray}{0.9}

% Caio - tentando melhorar o posicionamento das imagens
\usepackage{float}

% Caio - corrigindo espaçamento conforme http://tex.stackexchange.com/questions/5683/how-to-remove-top-and-bottom-whitespace-of-longtable
\setlength{\LTpre}{0pt}
\setlength{\LTpost}{0pt}

% Caio - preciso de plotagens
%\usepackage{pgfplots}
%\pgfplotsset{compat=1.8}

% Caio - quero usar letras nas listas do enumerate conforme https://tex.stackexchange.com/questions/2291/how-do-i-change-the-enumerate-list-format-to-use-letters-instead-of-the-defaul
\usepackage{enumitem}

% Caio - modo paisagem para tabelões
\usepackage{lscape}

% Caio - adicionando o pacote hyperref
\usepackage{hyperref}
% - e definindo metadados do PDF e comportamento dos links
\hypersetup{
	%pagebackref=true,
	pdftitle={Trabalho de Introdução à Probabilidade e Esatística}, 
	pdfauthor={Vários},
	pdfsubject={estatística},
	colorlinks=false,      		% false: boxed links; true: colored links
	linkcolor=blue,          	% color of internal links
	citecolor=blue,        		% color of links to bibliography
	filecolor=magenta,      	% color of file links
	urlcolor=blue,
	bookmarksdepth=4
}

% Caio - separação silábica
%\hyphenation{}

% Caio - citações mais poderosas
%\usepackage[autostyle]{csquotes}

%-----------------------------------------------------------
%-----------------------------------------------------------
% Caio - habilitar glossário
\usepackage{glossaries}
\makeglossaries

% \newglossaryentry{ex}{name={sample},description={an example}}
\newglossaryentry{ufabc}{
	name={UFABC},
	description={Universidade Federal do ABC}
}

\newglossaryentry{ead}{
	name={EAD},
	description={Ensino à Distância}
}

\newglossaryentry{www}{
	name={WWW},
	description={World Wide Web}
}
%-----------------------------------------------------------
%-----------------------------------------------------------
% Comandos para definir ambientes tipo teorema em português 
\newtheorem{meuteorema}{Teorema}[chapter]
\newtheorem{meuaxioma}{Axioma}[chapter]
\newtheorem{meucorolario}{Corolário}[chapter]
\newtheorem{meulema}{Lema}[chapter]
\newtheorem{minhaproposicao}{Proposição}[chapter]
\newtheorem{minhadefinicao}{Definição}[chapter]
\newtheorem{meuexemplo}{Exemplo}[chapter]
\newtheorem{minhaobservacao}{Observação}[chapter]
%-----------------------------------------------------------
%-----------------------------------------------------------
% Pacotes de citações
\usepackage[brazilian,hyperpageref]{backref}
\usepackage[alf]{abntex2cite}   % Citações padrão ABNT
%\usepackage[num]{abntex2cite}  % Citações numéricas
% --- 
% Configurações do pacote backref
% Usado sem a opção hyperpageref de backref
\renewcommand{\backrefpagesname}{Citado na(s) página(s):~}
% Texto padrão antes do número das páginas
\renewcommand{\backref}{}
% Define os textos da citação
\renewcommand*{\backrefalt}[4]{
	\ifcase #1 %
	Nenhuma citação no texto.%
	\or
	Citado na página #2.%
	\else
	Citado #1 vezes nas páginas #2.%
	\fi}%
% --- 
% --- 
% Espaço em branco no início do parágrafo
\setlength{\parindent}{1.3cm}
% Controle do espaçamento entre um parágrafo e outro:
\setlength{\parskip}{0.2cm}  % tente também \onelineskip
% ---
% compila o indice, se este for incluído no texto
\makeindex
%
% --------------------------------------------------------- 
% ---------------------------------------------------------
% Redefinindo o comando do abntex2 para gerar uma capa  
\renewcommand{\imprimircapa}{%
	\begin{capa}
		\begin{figure}
			\centering
			\includegraphics[width=0.3\linewidth]{img/logotipo-ufabc-abaixo}
		\end{figure}	
		\begin{flushleft} 
			{\centering \Large \textsc{\imprimirinstituicao  \\
					\imprimircurso \\} }
		\end{flushleft}
		
		\vfill
		\begin{center}
			{\large \imprimirautor} \\
			\vspace*{0.5cm}
			{\Large \textit{\imprimirtitulo}}
		\end{center}
		
		\vfill
		\begin{center}
			{\large{\imprimirlocal \\ \imprimirano  }}
		\end{center}
		\vspace*{1cm} 
	\end{capa}
	
}

% ---------------------------------------------------------
% ---------------------------------------------------------
%
%
% ---------------------------------------------------------
% ---------------------------------------------------------
% Redefinindo o comando para gerar uma folha de rosto 
\renewcommand{\imprimirfolhaderosto}{%
	\begin{center}
		{\large \imprimirautor}
	\end{center}
	\vfill \vfill \vfill \vfill
	\begin{center}
		{\Large \textit{\imprimirtitulo}}
	\end{center}
	
	\vfill \vfill \vfill 
	\begin{flushright} 
		\parbox{0.5\linewidth}{
			\imprimirtipotrabalho\ do
			\imprimircurso\, da \imprimirsigla\, 
			entregue como parte da disciplina de Introdução à Probabilidade e Estatística.}
	\end{flushright} 
	
	\vfill 
	\begin{flushright} 
		\parbox{0.5\linewidth}{ \imprimirorientadorRotulo 
			\imprimirorientador\\ \imprimirttorientador}
	\end{flushright} 
	
	\ifdefvoid{\imprimircoorientador}{}{
		\begin{flushright} 
			\parbox{0.5\linewidth}{ \imprimircoorientadorRotulo 
				\imprimircoorientador\\ \imprimirttcoorientador}
		\end{flushright}
	}
	
	\vfill \vfill \vfill \vfill \vfill \vfill \vfill
	\begin{center}
		{\large{\imprimirlocal \\ \imprimirano}}
	\end{center}
	\vspace*{1cm} \newpage
}
% Final do comando para gerar uma folha de rosto 
% ---------------------------------------------------------
% ---------------------------------------------------------
%
%
% ---------------------------------------------------------
% ---------------------------------------------------------
% Definindo o comando para gerar uma folha de defesa 
\newcommand{\imprimirfolhadeaprovacao}{%
	\begin{center}
		{\large \imprimirautor}
	\end{center}
	\vfill \vfill \vfill \vfill
	\begin{center}
		{\Large \textit{\imprimirtitulo}}
	\end{center}
	
	\vfill \vfill \vfill \vfill \vfill \vfill
	\begin{flushright} 
		\parbox{0.5\linewidth}{
			%			\imprimirtipotrabalho\,apresentada ao 
			%			\imprimircurso\, da \imprimirsigla\, como requisito
			%			para a obtenção parcial do grau de \imprimirgrau.}
		}
	\end{flushright} 
	\vfill \vfill \vfill \vfill
	Aprovada em \data.
	
	\vfill \vfill \vfill \vfill
	
	\begin{center}
		\textbf{BANCA EXAMINADORA}
		
		\vfill\vfill\vfill
		\rule{10cm}{.1pt}\\
		{\imprimirexaminadorum} \\ {\imprimirttexaminadorum}
		
		\ifdefvoid{\imprimirexaminadordois}{}{
			\vfill\vfill
			\rule{10cm}{.1pt}\\
			\imprimirexaminadordois \\ \imprimirttexaminadordois }
		
		\ifdefvoid{\imprimirexaminadortres}{}{
			\vfill\vfill
			\rule{10cm}{.1pt}\\
			\imprimirexaminadortres \\ \imprimirttexaminadortres }
		
		\ifdefvoid{\imprimirexaminadorquatro}{}{
			\vfill\vfill
			\rule{10cm}{.1pt}\\
			\imprimirexaminadorquatro \\ \imprimirttexaminadorquatro }
	\end{center}
	
	\vfill \vfill 
	\begin{center}
		{\large{\imprimirlocal \\ \imprimirano}}
	\end{center}
	\vspace*{1cm}
	\newpage
}
% Final do comando para gerar uma folha de defesa 
% ---------------------------------------------------------
% --------------------------------------------------------
%
%
%
%
%
% ---------------------------------------------------------
% --------------------------------------------------------
% definindo variáveis adicionais 
\providecommand{\imprimirsigla}{}
\newcommand{\sigla}[1]{\renewcommand{\imprimirsigla}{#1}}
%
\providecommand{\imprimircurso}{}
\newcommand{\curso}[1]{\renewcommand{\imprimircurso}{#1}}
%
\providecommand{\imprimirano}{}
\newcommand{\ano}[1]{\renewcommand{\imprimirano}{#1}}
%
\providecommand{\imprimirgrau}{}
\newcommand{\grau}[1]{\renewcommand{\imprimirgrau}{#1}}
%
\providecommand{\imprimirexaminadorum}{}
\newcommand{\examinadorum}[1]{
	\renewcommand{\imprimirexaminadorum}{#1}}
%
\providecommand{\imprimirexaminadordois}{}
\newcommand{\examinadordois}[1]{
	\renewcommand{\imprimirexaminadordois}{#1}}
%
\providecommand{\imprimirexaminadortres}{}
\newcommand{\examinadortres}[1]{
	\renewcommand{\imprimirexaminadortres}{#1}}
%
\providecommand{\imprimirexaminadorquatro}{}
\newcommand{\examinadorquatro}[1]{
	\renewcommand{\imprimirexaminadorquatro}{#1}}
%
\providecommand{\imprimirttorientador}{}
\newcommand{\ttorientador}[1]{
	\renewcommand{\imprimirttorientador}{#1}} 
%
\providecommand{\imprimirttcoorientador}{}
\newcommand{\ttcoorientador}[1]{
	\renewcommand{\imprimirttcoorientador}{#1}}
%
\providecommand{\imprimirttexaminadorum}{}
\newcommand{\ttexaminadorum}[1]{
	\renewcommand{\imprimirttexaminadorum}{#1}}
%
\providecommand{\imprimirttexaminadordois}{}
\newcommand{\ttexaminadordois}[1]{\renewcommand{
		\imprimirttexaminadordois}{#1}}
%
\providecommand{\imprimirttexaminadortres}{}
\newcommand{\ttexaminadortres}[1]{
	\renewcommand{\imprimirttexaminadortres}{#1}}
%
\providecommand{\imprimirttexaminadorquatro}{}
\newcommand{\ttexaminadorquatro}[1]{
	\renewcommand{\imprimirttexaminadorquatro}{#1}}
% fim da definição de variáveis adicionais
% ---------------------------------------------------------
% ---------------------------------------------------------
%
% ---
% ---
% ---
% ---
% ---
% ---
% ---
% ---
% ---
% Informações de dados para CAPA, FOLHA DE ROSTO e FOLHA DE DEFESA
%
%----------------- Título e Dados do Autor -----------------
\titulo{Trabalho de Introdução à Probabilidade e Estatística}
\autor{Caio César Carvalho Ortega \and Carolina Horta Cattaneo}
%

%----------Informações sobre a Instituição e curso -----------------
\instituicao{Universidade Federal do ABC \\
	Centro de Matemática, Computação e Cognição}
%
\sigla{UFABC}
%
\curso{Bacharelado em Ciências e Humanidades}
%\curso{Curso de Licenciatura em Matemática}
%\curso{Mestrado em Ensino de Matemática}
%\curso{Doutorado em Matemática}
%
\local{São Bernardo do Campo, SP}
%
%
% -------- Informações sobre o tipo de documento
\tipotrabalho{Trabalho}
%\tipotrabalho{Monografia de final de curso}
%\tipotrabalho{Dissertação de mestrado}
%\tipotrabalho{Tese de doutorado}
%
\grau{BACHAREL em Ciências e Humanidades}
%\grau{LICENCIADO em Matemática}
%\grau{MESTRE em Matemática}
%\grau{DOUTOR em Ciências}
%
\ano{2019}
\data{20 de Agosto de 2019} % data da aprovação
%
%------Nomes do Orientador, examinadores.  
\orientador{ Prof. Dr. Antonio Sergio Munhoz}
%\coorientador{Antonio da Silva} % opcional
%\examinadorum{Dra. Kátia Canil}
%\examinadordois{Professor Dr. Fernando Rocha}
%\examinadortres{Jeferson Leandro Garcia de Araújo}
%\examinadorquatro{Antonio da Silva}
%
%--------- Títulos do Orientador e examinadores ----
%\ttorientador{Bacharel em Física - UEFS}
%\ttcoorientador{Doutor em Matemática - UFRJ} 
%\ttexaminadorum{Doutor em Matemática - UFRJ}
%\ttexaminadordois{Doutor em Matemática - UFRJ}
%\ttexaminadortres{Doutor em Matemática - UFRJ}
%\ttexaminadorquatro{Doutor em Matemática - UFRJ}
%
% ---
% ---
\begin{document}
	
	% ---
	% Chamando o comando para imprimir a capa
	\imprimircapa
	% ---
	% ---
	% Chamando o comando para imprimir a folha de rosto
	\imprimirfolhaderosto
	% ---
	% ---
	% Chamando o comando para imprimir a folha de aprovação
	%\imprimirfolhadeaprovacao
	% ---
	% ---
	% Dedicatória
	% ---
	%	\begin{dedicatoria}
	%  	 \vspace*{\fill}
	%  	 \centering
	%  	 \noindent
	%  	 \textit{ Este trabalho é dedicado a todos que, com entusiasmo,\\
	%  	 		sonham e lutam por XYZ no ABCDEFG\\
	%  			do XPTO.} \vspace*{\fill}
	%	\end{dedicatoria}
	%	
	%	
	%	\begin{agradecimentos}
	%	Orientação do modelo: insira aqui um parágrafo
	%	\end{agradecimentos}
	%	
	%	
	%
	%---------------------- EPÍGRAFE I (OPCIONAL)--------------
	%\begin{epigrafe}
	%    \vspace*{\fill}
	%    \begin{flushright}
	%        \textit{''Texto''\\
	%        Autor}
	%    \end{flushright}
	%\end{epigrafe}
	%
	%
	%
	%--------Digite aqui o seu resumo em %Português--------------
	%\begin{resumo}
	%   Descrição. 
	%
	%   \vspace{\onelineskip}
	%   \noindent
	%   \textbf{Palavras-chaves}: Palavras.
	%\end{resumo}
	
	
	%
	% --- resumo em inglês (abstract) ---
	%\begin{resumo}[Abstract]
	%   \begin{otherlanguage*}{english}
	%      Description.
	%
	%      \vspace{\onelineskip}
	%      \noindent
	%      \textbf{Keywords}: Words.
	%   \end{otherlanguage*}
	%\end{resumo}
	
	%
	%----Sumário, lista de figuras e lista de tabelas ------------
	\tableofcontents 
%	\newpage \listoffigures
%	\newpage \listoftables
	%---------------------
	%--------------Início do Conteúdo---------------------------
	% o comando textual é obrigatório e marca o ponto onde começa 
	% a imprimir o número da página
	\textual
	%
	%---------------------
	%
	
	
	%
	% O conteúdo começa pra valer a seguir
	%
	
	%
	%===============================================================================
	%
	
	\chapter{Introdução}
	
	Este trabalho se propõe a aplicar os conhecimentos discutidos ao longo do curso, expandindo uma das primeiras atividades aplicadas, que consistiu na coleta, tratamento e análise de dados do site InfoJobs\footnote{\url{https://www.infojobs.com.br/mercado-livre/vagas}} para a empresa MercadoLivre.
	
	A partir de uma amostra aleatória contendo 100 mensagens deixadas no site (ou seja, 100 amostras), estas foram disponibilizadas \textit{online} por meio do \textit{software} Google Sheets, além disso, foram elaborados gráficos, tais como um histograma.
	
	Finalmente, a estratégia adotada para o MercadoLivre foi adaptada para a \gls{ufabc}, utilizando dados da plataforma UFABC Next\footnote{\url{https://ufabcnext.com}}, que suplantou o antigo UFABC Help!\footnote{\url{https://www.ufabchelp.me/}}, parametrizando valores com o intuito de proporcionar a melhoria dos planos de ensinos das disciplinas observadas.
	
	\chapter{Considerações metodológicas}
	
	\section{Intervalo de confiança}
	
	O primeiro aspecto metodológico a merecer destaque é a adoção da Regra dos 100, que nada mais é do que o estabelecimento de um intervalo de confiança para 100 amostras. Considerando uma população de 200, seria possível atingir um intervalo de confiança de 93,43\% com erro de 11,24\%. A exemplo da Regra dos 50 fornecida nas especificações do trabalho\footnote{\url{https://docs.google.com/spreadsheets/d/1o_8Owjhj2WIb7zI8xsssiL_a5Xm_HvKiqtb8Aqjy3Hg/edit?usp=sharing}}, foi elaborada planilha análoga para a Regra dos 100 utilizando o \textit{software} LibreOffice.org Calc, com a posterior carga e conversão no disco virtual do Google (Google Drive), permitindo acesso remoto por meio do \textit{software} Google Sheets, por meio do endereço \url{https://docs.google.com/spreadsheets/d/1EVE47HOMK7rcjOnfLdPdV3hR6_NcSbeoKX3M4Z51jJk/edit?usp=sharing}. A Regra dos 100 foi baseada na lógica da Regra dos 50 fornecida no escopo do trabalho disponibilizado eletronicamente na plataforma de \gls{ead} Tidia da \gls{ufabc} por meio do endereço \url{https://docs.google.com/spreadsheets/d/1o_8Owjhj2WIb7zI8xsssiL_a5Xm_HvKiqtb8Aqjy3Hg/edit?usp=sharing}.
	
	As planilhas estão construídas adotando o conceito de distribuição hipergeométrica, adequada para extrações casuais de uma população que são feitas sem reposição, dividida em dois atributos. O termo \textit{split} (divisão, traduzindo livremente do inglês para o português brasileiro) está presente nas planilhas em virtude dessa divisão. A distribuição hipergeométrica é conceituada por \citeonline[p. 147]{morettin2017} como:
	
	\begin{citacao}
	    Essa distribuição é adequada quando consideramos extrações casuais feitas sem reposição de uma população dividida segundo dois atributos. Para ilustrar, considere uma população de $N$ objetos, $r$ dos quais têm o atributo $A$ e $N - r$ têm o atributo $B$. Um grupo de n elementos é escolhido ao acaso, sem reposição. Estamos interessados em calcular a probabilidade de que esse grupo contenha $k$ elementos com o atributo $A$.
	\end{citacao}
	
	O cálculo da probabilidade mencionada por \citeonline{morettin2017} é dado pela equação a seguir, utilizando o princípio multiplicativo:
	
	\begin{equation}
	    p_k = \dfrac{\begin{pmatrix}
	r \\
	k
	\end{pmatrix}\begin{pmatrix}
	N - r \\
	n - k
	\end{pmatrix}}{\begin{pmatrix}
	N \\
	n
	\end{pmatrix}}
	\end{equation}
	
	Finalmente, conforme \citeonline[p. 147]{morettin2017}, ``os pares $(k, p_k)$ constituem a distribuição hipergeométrica de probabilidades'', sendo a variável aleatória $X$ ``o número de elementos na amostra que têm o atributo $A$, então $P(X = k) = p_k$''.
	
	Como aponta \citeonline[p. 45]{field2009}, o valor do intervalo de confiança se traduz na ideia de que ele representa quantos intervalos possuem o valor real da média da população, em outras palavras, o valor indica a confiabilidade da estimativa:
	
	\begin{citacao}
		Tipicamente, se prestarmos atenção aos intervalos de confiança de 95\% e, algumas	vezes, aos intervalos de confiança de 99\%, veremos que eles têm interpretações semelhantes: são limites construídos para que em certa percentagem das vezes (seja 95\% ou 99\%) o	valor real da média da população esteja dentro desses limites. Assim, quando você tiver um intervalo de confiança de 95\% para uma média, pense nele assim: se selecionarmos 100 amostras, calcularmos a média e, depois, determinarmos o intervalo de confiança para aquela média (\dots), 95\% dos intervalos de confiança conterão o valor real da média da população.
	\end{citacao}
	
	Na abordagem do tema feita por \citeonline[p. 45-47]{field2009}, o cálculo do intervalo de confiança está intimamente ligado com os escores-z, porque um intervalo de confiança de 95\%, por exemplo, corresponde ao valor de $z$ de $1,96$, colocando os limites do intervalo de confiança em $-1,96$ e $+1,96$, correspondendo, portanto, a uma distribuição normal com média $0$ e desvio padrão $1$ \cite[p. 45]{field2009}. Sabendo-se que a conversão de escores em escores-z é feita por meio da \autoref{eq:escoresz}, sendo $z$ o escore-z, $X$ o valor, $\overline{X}$ a média e $s$ o desvio padrão \cite[p. 41]{field2009}.
	
	\begin{equation}
	    \label{eq:escoresz}
	    z = \dfrac{X - \overline{X}}{s}
	\end{equation}
	
	Sabendo que a média está sempre no intervalo de confiança, o cálculo deste é feito reorganizando as equações, mantendo a mesma lógica empregada anteriormente \cite[p. 47]{field2009}.
	
	\begin{equation}
	    z \times s = X - \overline{X}
	\end{equation}
	
	\begin{equation}
	    z \times s + \overline{X} = X
	\end{equation}
	
	\section{Captura e organização dos dados}

	A coleta/captura dos dados ocorreu sem o emprego de técnicas de programação ou de automação para fazer o \textit{scrapping} dos dados\footnote{A terminologia pode ser traduzida livremente para o português brasileiro como ``raspagem'' e é empregada em outros artigos, como em \citeonline[p. 212]{limajr2012}}. Os passos adotados assim podem ser sumarizados:
	
	\begin{enumerate}
	    \item Abertura do endereço desejado da \gls{www} no navegador;
	    \item Identificação das variáveis desejadas;
	    \item Estruturação de uma planilha capaz de acomodar os dados das variáveis verticalmente, sendo uma variável por coluna, um valor célula, de maneira que para a variável $V_1$ na coluna $A$, possuindo esta três valores, estes serão acomodados nas células $A2$, $A3$ e $A4$, ficando a célula $A1$ reservada para o título ou nome da variável;
	    \item Preenchimento da planilha, transferindo os dados por mecanismos de cópia e colagem nativos do sistema operacional.
	\end{enumerate}

    A classificação das variáveis adotou a conceituação realizada por \citeonline[p. 6]{pinheiro2009}:
	
	\begin{citacao}
		\textbf{Variável qualitativa nominal} ou \textbf{categórica} --- seus valores possíveis são diferentes categorias não-ordenadas, em que cada observação pode ser classificada. Exemplos: raça, nacionalidade, área de atividade. \\
		\textbf{Variável qualitativa ordinal} --- seus valores possíveis são diferentes categorias ordenadas, em que cada observação pode ser classificada. Exemplos: classe social, nível de instrução. \\
		\textbf{Variável quantitativa discreta} --- seus valores possíveis são em geral resultados de um processo de contagem. Exemplos: número de filhos, número de séries escolares cursadas com aprovação. \\
		\textbf{Variável quantitativa contínua} --- seus valores podem ser expressos através e números reais e varrem uma escala contínua de medição. Exemplos: renda mensal, peso, altura.
	\end{citacao}
	
	\subsection{Dados do InfoJobs}
	
	A \autoref{tab:variaveisml} classifica as variáveis para a amostra de $100$ avaliações do site InfoJobs para a empresa MercadoLivre. Foram coletadas as avaliações feitas pelos usuários do site no período entre 18 de Fevereiro de 2019 e 30 de Maio de 2019.
	
    A fim de que a amostra tivesse resultados mais homogêneos e relevantes, os dados foram previamente filtrados no próprio site pela localidade, fazendo com que fossem apresentados apenas avaliações do estado de São Paulo. 
	
	\begin{center}
    \begin{longtable}{|l|l|l|}
    \caption{Tipos de variáveis para o MercadoLivre} \label{tab:variaveisml} \\
    
    \hline \multicolumn{1}{|c|}{\textbf{Nome da variável}} & \multicolumn{1}{c|}{\textbf{Tipo de variável}} & \multicolumn{1}{c|}{\textbf{Subtipo de variável}} \\ \hline 
    \endfirsthead
    
    \multicolumn{3}{c}%
    {{\bfseries \tablename\ \thetable{} -- continuado da página anterior}} \\
    \hline \multicolumn{1}{|c|}{\textbf{Nome da variável}} & \multicolumn{1}{c|}{\textbf{Tipo de variável}} & \multicolumn{1}{c|}{\textbf{Subtipo de variável}} \\ \hline 
    \endhead
    
    \hline \multicolumn{3}{r}{{Continua na próxima página}} \\
    \endfoot
    
    \hline \hline
    \endlastfoot
    
    Estrelas & Quantitativa & Discreta \\
    Global & Quantitativa & Discreta \\
    Oportunidade de promoção & Quantitativa & Discreta \\
    Ambiente de trabalho & Quantitativa & Discreta \\
    Conciliação com a vida familiar & Quantitativa & Discreta \\
    Benefícios & Quantitativa & Discreta \\
    Recomenda a Empresa a um amigo & Qualitativa & Nominal \\
    Aprova a Diretoria & Qualitativa & Nominal \\
    Funcionário/Ex sim/não & Qualitativa & Nominal \\
    Data & Qualitativa & Ordinal \\
    Cargo & Qualitativa & Nominal \\
    Comentário & Qualitativa & Nominal \\
    Prós & Qualitativa & Nominal \\
    Contras & Qualitativa & Nominal \\
    Dica a Diretoria & Qualitativa & Nominal \\
    \end{longtable}
    \end{center}
    
    \subsection{Dados do UFABC Next}
    
    No caso do MercadoLivre, a obtenção das $100$ amostras não significou grandes entraves, por outro lado, a obtenção de avaliações de alunos do sistema UFABC Next exigiu a observação de mais de um docente, mais precisamente, de $19$ (dezenove) professores:
    
    \begin{itemize}
        \item A. M. Timpanaro;
        \item A. M. Veneziani;
        \item A. Magalhães;
        \item A. P. de Oliveira Junior;
        \item A. S. Munhoz;
        \item B. Marin;
        \item C. S. dos Santos;
        \item E. Alejandra;
        \item Ignat F.;
        \item L. A. da Silva;
        \item P. J. P. Martinez;
        \item P. M E. Claessens;
        \item R. H. A. H Jacobs;
        \item R. M. Coutinho;
        \item R. Venegeroles;
        \item S. Camargo;
        \item T. L. Ritchie;
        \item V. Marvulle;
        \item V. Perchine.
    \end{itemize}
	
				Comentário P/N	Didática P/N
	
		\begin{center}
		\begin{longtable}{|l|l|l|}
			\caption{Tipos de variáveis para o MercadoLivre} \label{tab:variaveisml} \\
			
			\hline \multicolumn{1}{|c|}{\textbf{Nome da variável}} & \multicolumn{1}{c|}{\textbf{Tipo de variável}} & \multicolumn{1}{c|}{\textbf{Subtipo de variável}} \\ \hline 
			\endfirsthead
			
			\multicolumn{3}{c}%
			{{\bfseries \tablename\ \thetable{} -- continuado da página anterior}} \\
			\hline \multicolumn{1}{|c|}{\textbf{Nome da variável}} & \multicolumn{1}{c|}{\textbf{Tipo de variável}} & \multicolumn{1}{c|}{\textbf{Subtipo de variável}} \\ \hline 
			\endhead
			
			\hline \multicolumn{3}{r}{{Continua na próxima página}} \\
			\endfoot
			
			\hline \hline
			\endlastfoot
			
			Professor & Qualitativa & Nominal \\
			Cobra Presença & Quantitativa & Discreta \\
			Conceito & Qualitativa & Ordinal \\
			Comentário P/N & Qualitativa & Nominal \\
			Didática P/N & Qualitativa & Nominal \\
		\end{longtable}
	\end{center}

	\chapter{Análise dos dados}
	% Responde itens 3, 4 e 5 e 7
	
	Como veremos a seguir, foram realizados três análises: (i) teste de aleatoriedade com determinado número de parâmetros para cada amostra; (ii) probabilidade de obtenção de determinado valor para determinada variável; e (iii) elaboração de tabelas de frequência e histogramas em relação aos parâmetros, observando o formato da curva; seleção de outro parâmetro binominal para verificar sua significância.
	
	Os análises foram realizadas para as duas amostras, uma vez que este trabalho optou por ir além dos dados do InfoJobs.
	
	\section{Dados do InfoJobs}
	
	A elaboração das tabelas de frequências para as variáveis contínuas discretas
	
	\section{Dados do UFABC Next}
	
	\chapter{Conclusão}


	%===============================================================================
	%
	
	% ----------------------------------------------------------
	% ----------------------------------------------------------
	\postextual
	
	
	
	% informa o arquivo com a bibliografia. Deve ser o mesmo nome
	% (sem o sufixo) que será informado no ambiente filecontents
	% que está no final deste arquivo. Neste exemplo foi usado 
	% bibitemp.bib e bibtemp. Este comando insere a bibliografia
	% nesta posição (antes dos apêndices, anexos, índice remissivo)
	\bibliography{fontes}
	% ----------------------------------------------------------
	% Glossário
	% ----------------------------------------------------------
	% Consultar manual da classe abntex2 para orientações sobre o
	% uso do glossário.
	\renewcommand{\glossaryname}{Glossário}
	%\renewcommand{\glossarypreamble}{Esta é a descrição do glossário.\\ \\}
	\renewcommand*{\glsseeformat}[3][\seename]{\textit{#1}
		\glsseelist{#2}}
	
	% ---
	% Traduções para o ambiente glossaries
	% ---
	\providetranslation{Glossary}{Glossário}
	\providetranslation{Acronyms}{Siglas}
	\providetranslation{Notation (glossaries)}{Notação}
	\providetranslation{Description (glossaries)}{Descrição}
	\providetranslation{Symbol (glossaries)}{Símbolo}
	\providetranslation{Page List (glossaries)}{Lista de Páginas}
	\providetranslation{Symbols (glossaries)}{Símbolos}
	\providetranslation{Numbers (glossaries)}{Números} 
	% ---
	
	% ---
	% Imprime o glossário
	% ---
	\cleardoublepage
	\phantomsection
	\addcontentsline{toc}{chapter}{\glossaryname}
	% \glossarystyle{index}
	% \glossarystyle{altlisthypergroup}
	\glossarystyle{tree}
	\printglossaries
	% ---
	
	% ----------------------------------------------------------
	% Apêndices
	% ----------------------------------------------------------
	
	% ---
	% Inicia os apêndices. Não esquecer de fechar ao final de
	% todos os apêndices (\end{apendicesenv})
	% ---
	%\begin{apendicesenv}
	
	% Imprime uma página indicando o início dos apêndices
	%\partapendices
	
	% ----------------------------------------------------------
	%\chapter{Primeiro apêndice}
	% ----------------------------------------------------------
	
	%Este é um exemplo de inclusão de capítulos de %apêndice em uma 
	%monografia.  Cada apêndice é tratado como se fosse %um capítulo.
	%Os apêndices devem ser iniciados pelo comando de %ambiente
	%\textbackslash begin\{apendicesenv\} e encerrados %pelo comando 
	%\textbackslash end\{apendicesenv\}.
	
	% ----------------------------------------------------------
	%\chapter{Segundo apêndice}
	% ----------------------------------------------------------
	
	%Este é um exemplo de inclusão de um segundo apêndice. 
	
	%\end{apendicesenv}
	% ---
	
	
	% ----------------------------------------------------------
	% Anexos
	% ----------------------------------------------------------
	
	% ---
	% Inicia os anexos
	% ---
	%\begin{anexosenv}
	
	% Imprime uma página indicando o início dos anexos
	%\partanexos
	
	% ---
	%\chapter{Anexo I}
	% ---
	%Os anexos são similares aos apêndices se distinguindo pelo fato
	%que os apêndices são de autoria do autor da monografia e os 
	%anexos não são da autoria do autor da monografia.  Por exemplo,
	%se incluir no trabalho um modelo de um formulário preenchido
	%por alunos participantes de uma pesquisa, este será um apêndice
	%se o formulário foi criado pelo autor da monografia e será um
	%anexo se o formulário tiver sido criado por outros (por exemplo,
	%é um formulário padrão da escola em que o aluno que o preenche
	%estuda).
	%
	%Mesmo que o formulário tenha sido elaborado pela escola, uma
	%reprodução do formulário preenchido por cada aluno na pesquisa
	%será incluído no apêndice pois envolve o trabalho do autor da
	%monografia ao distribuir, coletar e reproduzir as respostas.
	%
	%Este é um exemplo de inclusão de capítulos de anexos em uma 
	%monografia.  Cada anexo é tratado como se fosse um capítulo.
	%Os anexos devem ser iniciados pelo comando de ambiente
	%\textbackslash begin\{anexoenv\} e encerrados pelo comando 
	%\textbackslash end\{anexoenv\}.
	%
	%\end{anexosenv}
	% ---
	%---------------------------------------------------------------------
	%---------------------------------------------------------------------
	
	%\printindex
	
	% Por padrão são incluídas no trabalho somente as referências
	% citadas ao longo do texto. No comando abaixo foram acrescentadas
	% algumas referências não citadas (neste texto servem apenas como
	% exemplos). Não deve ser usado o comando (mais simples) 
	% \nocite{*}, pois este parece não ser compatível com o
	% abntex2cite
	%\nocite{abntex2cite,abntex2wiki,boyer,eves,iezzi,kletenic,
	%        diomara,steinbruch,intusolatex,feynman,shannon,
	%        luisfelipe,turing}
\end{document}
